\newpage
\subsubsection{Flight Report: After Crash}
\begin{minipage}{1\textwidth}
	\begin{flushright}
		Date: 05-06-2022\\
		Location: Near Sunrise Riverside Block K\\
		Coordinates: 10.724144, 106.704113\\
		Flight Mode: \gls{GPS}\\
		Flight Duration: 30 minutes\\\vspace{5mm}
	\end{flushright}
\end{minipage}

The goal of this flight was to test the drone after the crash that occurred. The drone had minor damages on the four rubber feet and external damage to the propeller mount.
The sensor enclosure of the drone was removed and the drone had thorough cleaning. The mounting frame remained on the drone. Calibration on the IMU, gyroscope, and compass was done again as instructed in the manual after a crash. \cite{splashdronemanual}

The drone flew for 30 minutes around a grass field at different speeds and altitudes. In the following sections, potential risks will be elaborated and evaluated based on this test flight.

\paragraph{The drone's feet were not stable enough to take off}
Due to the damage on the drone's feet, it might not be stable enough to take off. During this test flight, this thankfully did not occur. The drone could take off and land without any issues.
\paragraph{The damaged propeller mount did not secure the propeller}
Due to the crack of the damaged propeller mount, it could happen that the propeller wasn't secured enough during flight. Throughout manual testing and flight testing, this appeared to be no problem.


\paragraph{Conclusion}
While there is damage to the drone, it does not seem to hinder regular flights. 