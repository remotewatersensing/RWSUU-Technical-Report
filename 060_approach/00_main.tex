\newpage
\subsection{Approach} \label{approach}
The project consisted of several sprints. These sprints had time boxed periods, and usually lasted two to four weeks. Sprints can intertwine with each other if necessary. Each sprint is managed using a kanban \cite{kanban} board with three columns: To-Do, In-Progress and Done. These sprints are heavily influenced by competences given by the project assignment. \cite{assignmentform}

\subsubsection{Analyze} \label{sprint:analyze}
Existing water sensoring solutions and suitable drones were analyzed. Water quality standards used were also analyzed. Literature was be reviewed, and findings of which were contained in a preliminary research report (see appendix A). It is important to note that this document was extended during other sprints, when new info related to the project was found. A change log as well as the git version control system \cite{git} kept track whenever this document is edited.

\subsubsection{Design} \label{sprint:design}
An overview of the working architecture was designed, along with several prototypes of ways to measure the water quality using drones. The result of which is contained in the design section of the technical report.

To begin, effective variables to measure water quality were be selected. Sensors were selected to measure these variables based on their accuracy and capability to mount on a drone. A suitable drone was chosen using a morphological overview. Configurations of these sensors on the drone were designed and illustrated. 

\subsubsection{Realise} \label{sprint:realise}
Multiple prototypes were realized within the given time and budget. Mounting was made. Software was written to fly the drone and to transmit sensor data. This software includes documentation befitting industry standards.

\subsubsection{Research} \label{sprint:research}
From each of these prototypes, several quantitative and qualitative features were be tested. These features include accuracy, speed, and cost. The summary of this descriptive, non-probabilistic research is written down in \ref{testing}

\subsubsection{Professionalise} \label{sprint:professionalise}
The most compelling prototype was be refined to a minimum viable product. \cite{mvp} This \gls{MVP} can be derived from one prototype or mixed together from multiple prototypes. The process will be similar to the realize sprint \ref{sprint:realise}, with the final design being added in the design section \ref{sprint:design}
