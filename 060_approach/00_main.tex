\newpage
\section{Approach} \label{approach}
The project consist of several sprints. These sprints will have time boxed periods, and usually last two to four weeks. Sprints can intertwine with each other if necessary. Each sprint is managed using a kanban \cite{kanban} board with three columns: To-Do, In-Progress and Done. These sprints are heavily influenced by competences given by the project assignment. \cite{assignmentform}

\subsection{Analyze} \label{sprint:analyze}
Existing water sensoring solutions and suitable drones are analyzed. Water quality standards used are also analyzed. Literature will be reviewed, and findings of which are contained in a preliminary research report (see appendix A). It is important to note that this document can be extended during other sprints, when new info related to the project is found. A change log as well as the git version control system \cite{git} will keep track whenever this document is edited.

\subsection{Design} \label{sprint:design}
An overview of the working architecture is designed, along with several prototypes of ways to measure the water quality using drones. The result of which is contained in the design section of the technical report.

To begin, effective variables to measure water quality will be selected. Sensors will be selected to measure these variables based on their accuracy and capability to mount on a drone. A suitable drone will be chosen using a morphological overview. Configurations of these sensors on the drone will be designed and illustrated. 

\subsection{Realise} \label{sprint:realise}
Multiple prototypes will be realized within the given time and budget. Mounting will be made if necessary. Software will be written to fly the drone and to read (and possibly transmit) sensor data. This software will include documentation befitting industry standards.\\

During this sprint, no new reports will be written. Whenever there is an issue with a design, the design report can be edited. A change log as well as git will keep track of these edits.

\subsection{Research} \label{sprint:research}
From each of these prototypes, several quantitative and qualitative features will be tested. These features include accuracy, speed, and cost. The summary of this descriptive, non-probabilistic research is written down in the technical report

\subsection{Professionalise} \label{sprint:professionalise}
If there is time left, the most compelling prototype will be refined to a minimum viable product. \cite{mvp} This \gls{MVP} can be derived from one prototype or mixed together from multiple prototypes. The process will be similar to the realize sprint \ref{sprint:realise}, with the final design being added in the design report.
