\newpage
\section{Conclusion}
The purpose of this project was to find out what the best way is to build a mobile surface water quality sensoring system that can take samples in remote areas. Based on the preliminary research, design, and testing that has been done during this project, it can be concluded that there are multiple options for such a system to be realized. \\

The electronic sensors used were accurate enough to prove useful in a real use case. Furthermore, the software that has been written proved to be working as expected and up to standards. It turns out it is possible to make a remote water sensing system using drones from a software perspective.\\

The sensor package leaves much to be desired from a design standpoint. From testing, it turned out that the drone with sensor package is too unstable to perform any measurements at this point in time. 

As seen, turbidity measurements with a secchi disk turns out to have potential, though the weight of such a disk causes the drone to use more power, which in turn reduces the battery life, limiting real implementations.

With the multidisciplinary research that has been done, it can be concluded that an improvement of the current sensor package design would be the best way to build a mobile surface water quality sensoring system that can take samples in remote areas. While a minimum viable product wasn't realized, the project did succeed to adhere to the rest of the MoSCoW prioritizations.

\subsection{Recommendations}
This project has come a long way in realizing a remote water sensing solution using drones. There are a few things that still need to be done to make this project a success.\\

A recommendation is to decrease the size and weight of the sensor package. An electronics engineering student could work on creating a custom printed circuit board, and a mechatronics student could for example work on designing a more permanent mounting solution. More care has to be put into the mounting location of the sensor package, so that it does not block the compass.\\

If the drone is stable enough, one could start to work on some automation by controlling the drone via the flight control communication protocol \cite{splashdronemanual} rather than the app. If this is a success, machine vision could be used as a way to automate secchi disk measurements and to detect anomalies in the water.\\

During this project, near infrared cameras as mentioned in the preliminary research report weren't realized as there was a lack of time to really do this subject with the attention it deserves. A recommendation is to reconsider it as an alternative option to take remote samples.