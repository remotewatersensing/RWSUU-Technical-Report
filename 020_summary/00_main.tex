\newpage
\section*{Summary}
The Mekong Delta turns out to suffer from trends like salinization, subsidence, and desiccation. These trends are mostly caused by controllable influences such as hydro power installations and ground water extraction.\\

After some preliminary research and looking at related studies, a belief was established that a faster, autonomous, and cheaper way to monitor water quality changes in the Mekong Delta could be developed.\\

The aim of this project was to explore ways to measure water quality parameters using unmanned aerial vehicles in order to improve the water quality monitoring in an area like the Mekong Delta.\\

Relevant water quality parameters (turbidity, conductivity, and acidity) and their sensors were analyzed and selected. Software was written to log these sensor values both locally on the sensor package and remotely on a graphical web interface.\\

Two revisions were made on the sensor package hardware. After testing, it turned out that the first revision failed to be water proof, and the second revision blocked the compass on the drone, causing the drone to experience drifting and not be stable enough to perform real measurements.\\

The accuracy of the sensors were tested by comparing them to commercial sensors that were available. Accuracy requirements of conductivity, acidity, and temperature were provided by an interested party. The conductivity, acidity, and temperature sensors all passed these requirements with ease.\\

As seen, a fully working prototype was not far off with this project. Recommendations are to continue this project. A first step in this is to decrease the size and weight of the sensor package so that the drone will be stable enough to perform in-flight measurements.\\

Apart from the sensor package, a look has been taken at using a secchi disk with a drone to determine the turbidity of the water. After some testing, this method seem to have potential.