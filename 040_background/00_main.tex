\subsection{Background}
This project has been brought forward by an idea of the International Water Technology research group at the Saxion University of Applied Sciences. The project has been carried out in collaboration with PERNAM Joint Stock Group (\gls{JSC}) and Robor Electronics \cite{robor}. PERNAM provided a test site and test case, while Robor provided access to the drone and provided general UAV assistance. Ton Duc Thang University (\gls{TDTU}) further accommodated the project when possible by providing expert advice and practical necessities. As this project is a first of potential following projects, emphasis is laid on developing an open source and well documented foundation so that future parties can build on this project in the future. 

Completing this project is a start of providing smarter water sensing solutions which in turn could benefit current water filtration solutions of PERNAM as well as help with several water availability issues.

\subsubsection{PERNAM JSC}
PERNAM \gls{JSC} \cite{pernam} is an engineering firm specializing in water treatment solutions in Vietnam. They are specialized in the design, engineering, and construction supervision of water treatment technologies for groundwater, surface, and brackish water. The company partially originated from the Netherlands and has close connections with Vitens, the biggest water supply company in the Netherlands. PERNAM is closely related to Howaco, the only private water supply company who is member of the South Vietnam Water Supply and Sewerage Association (\gls{VWSA}).

\subsubsection{Ton Duc Thang University}
Ton Duc Thang University \gls{TDTU} \cite{tdtu} is a public research university with the main campus in Ho Chi Minh City, Vietnam. The University offers 40 undergraduate programs, 18 master programs and 25 doctoral programs in variety of areas such as: law, applied science, technology, vocational skills, social sciences, economics, business, foreign languages and arts. 
\newpage
\subsubsection{Mekong River}
This project is focused around the Mekong River. The Mekong River is a network of tributaries in southwest Vietnam, between Ho Chi Minh City and Cambodia. The river itself starts in the Himalayas and passes through China, Myanmar, Thailand and Cambodia before reaching Vietnam.

The Mekong River is generally regarded as vital to the Vietnamese economy: A fourth of Vietnam's total agricultural sector takes place in the Mekong River. It is also regarded as Vietnam's most important fishing region, contributing more than half of Vietnam's total fishery output \cite{vietstats}

The region's production capabilities is threatened by multiple external sources. Vietnam has little influence on the water levels of the Mekong River, because the amount of water flowing into the country is regulated by dams in China and Laos. The quality of the water also differs regularly because of these upstream influences. Climate change also plays a big part, with more rainy seasons and a rise of sea levels. \cite{wur}

Apart from external influences, the Mekong River is also rapidly losing elevation due to accelerating subsidence rates, primarily caused by increasing ground water extraction. This strongly increases the river’s vulnerability to flooding, salinization, and coastal erosion. \cite{minderhoud2020}